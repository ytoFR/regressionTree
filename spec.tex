


Fonctionnalités souhaitées :
\begin{itemize}
	\item Plusieurs réponses, une partie des coefficients commune aux différentes réponses
	\item Arbre de régression, transformation non linéaire d'une réponse utilisée comme variable dans le modèle d'une autre réponse
	\item Contraintes d'intervalles sur les coefficients
	\item Contrainte sur la somme des coefficients de chaque modèle pour sélectionner les variables
	\item Contrainte de monotonie pour une réponse par rapport à une variable identifiée
	\item Contrainte d'extremum d'une réponse dans un domaine identifié
	\item Contrainte d'ordre des estimations en relation avec l'ordre des observations et une valeur de dispersion connue
	\item Contraintes sur les estimations d'un ensemble d'observations pour lequel la réponse n'est pas connue
	\item Accepter les valeurs manquantes sur les facteurs et les réponses
	\item Poids sur les résidus (positifs ou nuls)
\end{itemize}



Arbres de régression sous contraintes

La recherche des arbres optimaux est effectuée en plusieurs étapes :
- créer les transformations possibles de réponses et variables
- créer les contraintes, décrire l'arbre recherché
- découper la recherche de l'arbre en un ensemble de jobs indépendants = le découpage donne des PLMIP. Par exemple si une formule non linéaire implique de rechercher une valeur par dichotomie alors on gère cette dichotomie par des jobs
- découper les essais en ensemble d'apprentissage et de test
- réaliser les jobs en // ou séquentiellement.

Critères
- moindres écarts
- moindres carrés
- erreur max
- critère d'erreur sur un arbre = somme des erreurs 
- critère de courbure minimale en interpolation pour gérer l'overfitting
- on peut avoir plusieurs critères pour une même réponse

Réponses
- une réponse est un scalaire ou un signal sur lequel on prélève des critères scalaires
- pour chaque réponse une liste de transformations
- des relations possibles entre les réponses = des équations dans un intervalle
- un ordre de modélisation entre les réponses = relations de cause à effet
- réponse fictive à construire et servant de variable cachée = tronc commun entre les réponses

Variables
- variables de base : scalaire ou signal sur lequel on prélève des critères
- transformations sur les variables de base
- équivalences entre les transformations de variables sur certains critères ? Par ex la carré et le sinus si 3 niveaux
- variables fictives communes à plusieurs réponses : on leur donne un nom et on les contraint avec une formule maxi sur une liste de variables de base

Contraintes
- sur estimations des essais réalisés
- sur estimation d'essais non réalisés
- sur les coefficients
- sur les résidus : erreur maxi en absolu ou en % de la réponse ou d'une variable
- sur la somme des coeffs en valeur absolue
- sur le max des coeff
- sur le nombre de coeff non nuls par modèle et par arbre
- ordre sur les estimations pour un ensemble d'essais
- ordre sur les estimations en fonction d'une clé de tri
- monotonie dans une direction quelconque
- sur la dérivée d'un modèle
- sur chaque réponse : combien de transformations peut utiliser l'arbre au maximum, combine de modèles par réponse
- ordre sur la précision des modèles entre différentes réponses
- critère d'erreur globale minimal quand on connait la répétabilité
- courbure minimale 

Algorithme
- procédure SEP sur les jobs.
- le job sans contrainte ou relaché peut servir de coupe pour les autres jobs.
- PL MIP NEOS ou CPLEX pour les jobs